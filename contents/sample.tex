\appchapter{对\LaTeX 不熟悉可以参考这里}\label{chap:sample}
\textbf{\color{blue}编号条目使用asparaenum编写较为紧凑美观:}
\begin{asparaenum}[1)]
	\item 固体氧化物燃料电池功重比低。
	\item 无涡轮燃烧室流场组织复杂、易出现燃烧不均匀。
\end{asparaenum}

\textbf{\color{blue}公式编写示例如下。}SOFC电池反应\cite{Fu2021}:
\begin{center}
	阳极:\ce{H_2 + O^2- -> H_2O + 2e^-};阴极:\ce{1/2O_2 + 2e^- -> O^2-}
\end{center}

电池中的电化学反应产生电能,电池电流即为电子传输速率,电流大小反映了电化学反应的剧烈程度,如式\eqref{icell}所示:
\begin{equation}
	i=\frac{dQ}{dt}=nF\frac{dN}{dt}=jA
	\label{icell}
\end{equation}

其中,$Q$为总电荷量,$n$为电子迁移量,$N$为物质的量,$j$为电流密度,$A$为反应面积。

多孔介质中气体流动并不缓慢因此不能直接使用Darcy扩散方程而要使用Brinkman方程\eqref{momentumGDE}\cite{Chellehbari2021,Celik2018}。
\begin{align}
	\nabla\left(\rho\textbf{u}\right)&=0\label{mass}\\
	\rho(\textbf{u}\nabla)\textbf{u}&=
	-\nabla p
	+\nabla\left[\mu\left(\nabla \textbf{u}+(\nabla \textbf{u})^{T}\right)-\frac{2\mu}{3}\left(\nabla{\textbf{u}}\right)\right]
	+\textbf{F}\label{momentum}\\
	\nabla(\rho{\textbf{u}})&=Q_{br}\label{massGDE}\\
	\frac{\rho}{\varepsilon^2}\left(\left(\textbf{u} \nabla\right)\textbf{u}\right)&={F}-\nabla{p}+
	\nabla\left[\frac{\mu}{\varepsilon}\left(\nabla{\textbf{u}}
	+\left(\nabla{\textbf{u}}\right)^{T}\right)
	-\frac{2\mu}{3\varepsilon}\left(\nabla{\textbf{u}}\right)\right]
	-\left(\frac{\mu}{\kappa}+
	\frac{Q_{br}}{\varepsilon^{2}}\right)\textbf{u}\label{momentumGDE}
\end{align}

\textbf{\color{blue} 各种图片的插入和引用格式,如图\ref{energypd}、图\ref{Yang2022}、图\ref{resultCCcloud}所示。目前使用[H]需要自行调节每个图片/表格的位置和大小以达到最佳的排版效果(也可以结合vspace\{\}或hspace\{\}指令调整间距),否则如果某页图片放不下则会跳转到下一页,原先位置出现大段留白。但是用htbp会导致图片固定在下一页页首(或b、p位置),后续段落被打断,部分文字前提填补上一页空白,剩下的部分在图片后。最好的方式是如果图片放不下就放在下一段结尾处(而不是页首)。目前还没有研究出自动化解决方案,如果有麻烦告诉我,wx:hello\_rua}
\begin{figure}[H]
	\centering
	\includegraphics[width=0.9\textwidth]{energypd.jpg}
	\caption{国际能源署2050净零方案关键里程碑图\cite{Chappell2021}}\label{energypd}
\end{figure}
\begin{figure}[H]
	\centering
	\subcaptionbox{同向流和逆向流}[14cm]{
		\includegraphics[width=0.75\textwidth]{direction1.png}
	}
	\subcaptionbox{交叉流}[14cm]{
		\includegraphics[width=0.75\textwidth]{direction2.png}
	}
	\caption{不同的流动方向及其电流密度分布图\cite{Yang2022}}\label{Yang2022}
\end{figure}
\begin{figure}[H]
	\centering
	\subcaptionbox{COMSOL出口速度云图}[6cm]{
		\includegraphics[height=3.5cm]{outCOMSOL.png}
	}
	\subcaptionbox{COMSOL横向速度云图}[6cm]{
		\includegraphics[height=3.5cm]{hCOMSOL.png}
	}
	\subcaptionbox{CFX出口速度云图}[6cm]{
		\includegraphics[height=3.5cm]{outCFX.png}
	}
	\subcaptionbox{CFX横向速度云图}[6cm]{
		\includegraphics[height=3.5cm]{hCFX.png}
	}
	\caption{计算结果对比图}\label{resultCCcloud}
\end{figure}
\textbf{\color{blue}表格格式如表\ref{Sasi2024}所示。}
\begin{table}[H]
	\centering
	\caption{氨氢物化性质表\cite{Sasi2024}}\label{Sasi2024}
	\begin{tabular}{ccc}
		\toprule%第一道横线
		&氨&氢\\
		\midrule%第二道横线
		密度(kg/m$^3$)&684&71.4\\
		沸点(K)&239.1&21\\
		分子质量(g/mol)&17&2\\
		低位热值(MJ/kg)&18.61&120\\
		比热容(kJ/kg$\cdot$K)&4.6&8.68\\
		燃点(K)&630&560\\
		\bottomrule%第三道横线
	\end{tabular}
\end{table}