% !TEX root = ../main.tex

\chapter{正文}

正文是主体,是作者对研究工作的详细表述。理(工)科类正文一般包括本研究内容的总体方案设计与选择论证,各部分(包括硬件与软件)的设计计算,试验方案设计的可行性、有效性以及试验(实验)数据处理及分析,理论分析等。管理人文类学科的论文一般包括对研究问题的论述及系统分析,比较研究,模型或方案设计,案例论证或实证分析,模型运行的结果分析或建议、改进措施等。应对本研究内容及成果进行较全面、客观的理论阐述,应着重指出本研究内容中的创新、改进与实际应用之处。凡引用他人观点、方案、资料、数据等,无论曾否发表,无论是纸质或电子版,均应详加注释。在科学研究和学术活动中的各种造假、抄袭、剽窃和其他违背科学共同体惯例的行为均属学术不端行为。
论文主体各章后应有一节“本章小结”。
\section{二级}
\subsection{三级}
\subsubsection{四级}
\indent
\section{本章小结}
本章干了啥?懒得看那么多正文,看个小结就够了。
\begin{asparaenum}[1)]
	\item 第一点
	\item 第二点
	\item 第三点
\end{asparaenum}


