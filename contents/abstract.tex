% !TEX root = ../main.tex

\begin{abstract}[zh]
\addcontentsline{toc}{chapter}{摘 \quad 要}
摘要又称内容提要,它应以浓缩的形式概括研究课题的内容、
方法和观点,以及取得的成果和结论,应能反映整个内容的精华。
中英文摘要以 300-500 字为宜(不超过 800 字)。撰写摘要时应
注意以下几点:
\begin{asparaenum}[1)]
\item 用精炼、概括的语言来表达,每项内容不宜展开论证或
说明;
\item 要客观陈述,不宜加主观评价;
\item 成果和结论性字句是摘要的重点,在文字论述上要多些,
以加深读者的印象;
\item 要独立成文,选词用语要避免与全文尤其是前言和结论
部分雷同;
\item 既要写得简短扼要,又要生动,在词语润色、表达方法
和章法结构上要尽可能写得有文彩,以唤起读者对全文阅读的
兴趣。
\end{asparaenum}
\end{abstract}

{
\newfontface{\arial}{Arial}[Scale=0.94]
\ctexset{chapter/format+={\arial}}
\begin{abstract}[en]
\addcontentsline{toc}{chapter}{ABSTRACT}
The abstract, also known as the summary of content, should summarize the content, methods and views of the research topic in a condensed form, as well as the results and conclusions obtained, which should reflect the essence of the entire content. The Chinese and English abstracts should be 300-500 words (not exceeding 800 words). When writing an abstract, the following points should be noted:
\begin{asparaenum}[1)]
	\item Express in concise and generalized language, and each item should not be elaborated or explained;
	\item Objective statements should be made, and subjective evaluations should not be added;
	\item The focus of the abstract should be on the results and conclusive sentences, and there should be more textual discussion to deepen the reader's impression;
	\item To be written independently, the choice of words and language should avoid similarity with the entire text, especially the introduction and conclusion parts;
	\item It should be written in a concise and lively manner, with emphasis on word refinement, expression methods, and organizational structure, in order to arouse readers' interest in reading the entire text.
\end{asparaenum}



\end{abstract}
}