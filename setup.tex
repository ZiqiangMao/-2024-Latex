% !TEX root = ./main.tex

\sjtusetup{
  %
  %******************************
  % 注意:
  %   1. 配置里面不要出现空行
  %   2. 不需要的配置信息可以删除
  %******************************
  %
  % 信息录入
  %
  info = {%
    %
    % 标题
    %
    zh / title           = {上海交通大学本科毕业论文\\2024修订版\LaTeX 模板},
    en / title           = {A \LaTeX \ Template for SJTU\\  Bachelor's Degree Thesis},
    %
    % 标题页标题
    %   可使用“\\”命令手动控制换行
    %
    % 关键词
    %
    zh / keywords        = {落叶的位置,谱出一首诗,时间在消逝,我们的故事开始},
    en / keywords        = {keyword1, keywoerd2,keyword3
    },
    % 姓名
    %
    zh / author          = {你的名字},
    en / author          = {Nide Name},
    %
    % 指导教师
    %
    zh / supervisor      = {某某教授},
    en / supervisor      = {Prof. Uom Uom},
    %
    % 副指导教师
    %
    % assoc-supervisor  = {某某教授},
    % assoc-supervisor* = {Prof. Uom Uom},
    %
    % 学号
    %
    id              = {XXXXXXXXXXXX},
    %
    % 学位
    %   本科生不需要填写
    %
    zh / degree          = {学士},
    en / degree          = {Bachelor of Engineering},
    %
    % 专业
    %
    zh / major           = {请输入你的专业名称},
    en / major           = {Your Major Here},
    %
    % 所属院系
    %
    zh / department      = {请输入你的学院名称},
    en / department      = {Your School Here},
    %
    % 答辩日期
    %   使用 ISO 格式 (yyyy-mm-dd);默认为当前时间
    %
     date                 = {202X-06},
    %
    % 标题页显示日期
    %   覆盖对应标题页的日期显示,原样输出
    %
     zh / display-date    = {202X 年 6 月},
     en / display-date    = {June, 202X},
    %
    % 资助基金
    %
    % zh / fund  = {
    %                {国家 973 项目 (No. 2025CB000000)},
    %                {国家自然科学基金 (No. 81120250000)},
    %              },
    % en / fund  = {
    %                {National Basic Research Program of China (Grant No. 2025CB000000)},
    %                {National Natural Science Foundation of China (Grant No. 81120250000)},
    %              },
  },
  %
  % 风格设置
  %
  style = {%
    %
    % 论文标题页 logo 颜色 (red/blue/black)
    %
    % title-logo-color = black,
  },
  %
  % 名称设置
  %
  name = {
    % bib             = {References},
    % ack             = {谢\hspace{\ccwd}辞},
    % achv            = {攻读学位期间完成的论文},
    digest={},
  },
}

% 使用 BibLaTeX 处理参考文献
%   biblatex-gb7714-2015 常用选项
%     gbnamefmt=lowercase     姓名大小写由输入信息确定
%     gbpub=false             禁用出版信息缺失处理
\usepackage[backend=biber,style=gb7714-2015,gbnamefmt=lowercase]{biblatex}
% 文献表字体
% \renewcommand{\bibfont}{\zihao{5}\fixedlineskip{15.6bp}}
% 文献表条目间的间距
\setlength{\bibitemsep}{0pt}
% 导入参考文献数据库
\addbibresource{refs.bib}

% 脚注格式
\usepackage[perpage,bottom,hang]{footmisc}

% 定义图片文件目录与扩展名
\graphicspath{{figures/}}
\DeclareGraphicsExtensions{.pdf,.eps,.png,.jpg,.jpeg}

% 确定浮动对象的位置,可以使用 [H],强制将浮动对象放到这里(可能效果很差)
\usepackage{float}

% 固定宽度的表格
% \usepackage{tabularx}

% 使用三线表:toprule,midrule,bottomrule。
\usepackage{booktabs}

% 表格中支持跨行
\usepackage{multirow}

% 表格中数字按小数点对齐
\usepackage{dcolumn}
\newcolumntype{d}[1]{D{.}{.}{#1}}

% 使用长表格
\usepackage{longtable}

% 附带脚注的表格
\usepackage{threeparttable}

% 附带脚注的长表格
\usepackage{threeparttablex}

% 算法环境宏包
\usepackage[ruled,vlined,linesnumbered]{algorithm2e}
% \usepackage{algorithm, algorithmicx, algpseudocode}

% 代码环境宏包
\usepackage{listings}
\lstdefinestyle{lstStyleCode}{%
  aboveskip         = \medskipamount,
  belowskip         = \medskipamount,
  basicstyle        = \ttfamily\zihao{6},
  commentstyle      = \slshape\color{black!60},
  stringstyle       = \color{green!40!black!100},
  keywordstyle      = \bfseries\color{blue!50!black},
  extendedchars     = false,
  upquote           = true,
  tabsize           = 2,
  showstringspaces  = false,
  xleftmargin       = 1em,
  xrightmargin      = 1em,
  breaklines        = false,
  framexleftmargin  = 1em,
  framexrightmargin = 1em,
  backgroundcolor   = \color{gray!10},
  columns           = flexible,
  keepspaces        = true,
  texcl             = true,
  mathescape        = true
}
\lstnewenvironment{codeblock}[1][]{%
  \lstset{style=lstStyleCode,#1}}{}

% 直立体数学符号
\providecommand{\dd}{\mathop{}\!\mathrm{d}}
\providecommand{\ee}{\mathrm{e}}
\providecommand{\ii}{\mathrm{i}}
\providecommand{\jj}{\mathrm{j}}

% 化学式
\usepackage[version=4]{mhchem}

% 国际单位制宏包
\usepackage{siunitx}

% 定理环境宏包
\usepackage{ntheorem}
% \usepackage{amsthm}

% 绘图宏包
\usepackage{tikz}
\usetikzlibrary{arrows.meta, shapes.geometric}

% 使用列表
\usepackage{paralist}

% 一些文档中用到的 logo
\usepackage{hologo}
\providecommand{\XeTeX}{\hologo{XeTeX}}
\providecommand{\BibLaTeX}{\textsc{Bib}\LaTeX}

% 借用 ltxdoc 里面的几个命令方便写文档
\DeclareRobustCommand\cs[1]{\texttt{\char`\\#1}}
\providecommand\pkg[1]{{\sffamily#1}}

% hyperref 宏包在最后调用
\usepackage[hidelinks]{hyperref}

% E-mail
\providecommand{\email}[1]{\href{mailto:#1}{\urlstyle{tt}\nolinkurl{#1}}}

%设置标注
\usepackage{caption}
\captionsetup[figure]{belowskip=0pt,aboveskip=1pt}
\captionsetup[table]{belowskip=0.5pt,aboveskip=1pt}

%插入外置原创性声明及使用授权书
\usepackage{pdfpages}

%注释
\usepackage{comment}

%改变字体颜色
\usepackage{color}